%%%sdocumentclass{book}
\documentclass{report}
\usepackage{graphicx}
\usepackage{url}
\begin{document}
\title{\huge \textbf{pipEdit} 
\break \break \small Version 0.8.5 \huge 
\break \break Users Guide \break \break \break}

%%%\begin{figure}
%%%\includegraphics[scale=0.5]{The_Pied_Piper_Logo.jpg}
%%%\caption{A boat.}
%%%\label{fig:boat1}
%%%\end{figure}
\author{The\_Piper \\ the\_piper@web.de}
\date{October 2020}
%%%\begin{document}
\maketitle
\tableofcontents
%%%\listoffigures
%%%\includegraphics[height=\baselineskip]{The_Pied_Piper_Logo.jpg}
\chapter{License}
This program is under the Gnu Public License, GPL.  \\ \break

Read the file \texttt{COPYING} or here:  \\ \break

\url{https://www.gnu.org/licenses/gpl-3.0.en.html}   \\ \break

for it‘s content. \\ \break

Excerpt: \\ \break

\textbf{    There is no warranty for the program, to the extent permitted by
	 applicable law.  Except when otherwise stated in writing the copyright
	 holders and/or other parties provide the program "as is" without warranty
	 of any kind, either expressed or implied, including, but not limited to,
	 the implied warranties of merchantability and fitness for a particular
	 purpose.  
	 \huge The entire risk as to the quality and performance of the program is with you. \normalsize
	 Should the program prove defective, you assume the cost of
	 all necessary servicing, repair or correction.
}

\chapter{Introduction}

pipEdit is meant to be a lookalike editor of the ISPF editor IBM offers on their mainframes, it is not meant to be a 100\% clone of it or of the whole ISPF environment. 


The plan is to create (and have) an editor which has a similar feeling like the IBM one, but some things might be different. 


For example, I haven‘t found out right now how to make a difference between the ENTER key and the ENTER key on the number pad, so one can be the XMIT/SEND key and the other won‘t.


That, and some other stuff, are some minor differences.


It can be compiled with ncurses and, if not available, with my tiny ncurses replacement, pipcurses, which should work with all vt52, vt100, ansi, xterm terminals.



\chapter{Compiling pipEdit}

pipEdit ist developed with gcc, the Gnu C compiler under Linux.

To compile it with ncurses type: \\

\texttt{gcc pipedit.c -opipedit.bin -O -lcurses} \break

Without ncurses, using my ncurses replacement, pipcurses, type this: \\

\texttt{gcc pipedit.c -opipedit.bin -O -D\_\_USE\_PIPCURSES\_\_=1} \\

Or use the Makefile, edit it to your needs and type \texttt{make} \\

Of course you can try to compile it for Windows, Dev-Cpp, for example, is a free compiler which should be able to compile source code for gcc. \\
 
\url{https://www.bloodshed.net/devcpp.html} \\

Or install GnuCobol for windows, which comes with a version of gcc for windows.
Here are the binaries for Windows: \\ \break
\url{https://arnoldtrembley.com/GnuCOBOL.htm#binaries}


\chapter{Configuring pipEdit}

pipEdit uses a configuration file to store/read it‘s configuration data, named \texttt{pipedit.cfg}.

\section{Environment variable}

To know where \texttt{pipedit.cfg} is located, the editor reads an environment 
variable named \texttt{\$PIPEDITCFG} which holds the complete path 
to the config file. \\ \break

Example:

\texttt{PIPEDITCFG="/home/myname/pipEdit/pipedit.cfg" ; export PIPEDITCFG} \break

Define this variable in \texttt{.bashrc}, \texttt{.kshrc} 
or whereever it suits you.

\section{Configuration file}

The configuration file itself looks like this:

\#\#\#\#\#\#\#\#\#\#\#\#\#\#\#\#\#\#\#\#\#\#\#\#\#\#\#\#\#\#\#\#\#\#\#\#

\# Global configuration for pipEdit \#

\#\#\#\#\#\#\#\#\#\#\#\#\#\#\#\#\#\#\#\#\#\#\#\#\#\#\#\#\#\#\#\#\#\#\#\#

\# target is used in *.par files to have multiple configurations  

\# for multiple compilers

\#

target=GNUCOBOL

\#\#\#target=MICROFOCUS

\#

\# Where are the macros located the edtior can use?

\#

macros=/home/myname/pipEdit/macros

\#

\# language overwrites the environment variable \$LANG

\# This can be used for language files for the editor, macros and 

\# so on.

\# If left blank, \$LANG is used, if \$LANG isn't defined "en" is

\# used.

\# default means, no file is read, the default values of the editor

\# are used.

\#

\#\#\#language=default

language=

\#

\# various translations of texts and messages

\#

langfiles=/home/myname/pipEdit/langfiles

\#

\# Function keys

\#

F1=HELP

F2=

F3=END

F4=

F5=RFIND

F6=

F7=UP

F8=DOWN

F9=

F10=LEFT

F11=RIGHT

F12=RETRIEVE

PGUP=UP

PGDOWN=DOWN

\chapter{Installing pipEdit}
To install pipEdit on a Linux system, follow these 3 easy steps: \\
\break

\textbf{One} \\
Copy the compiled program, \texttt{pipedit.bin}, to a directory which is 
specified in your \texttt{\$PATH}, like \texttt{/usr/local/bin}.

You might rename it to a shorter name, or create a shell script to call pipedit.bin, like \texttt{pe} or such. \\
\break

\textbf{Two} \\
Copy the config file \texttt{pipedit.cfg} to a location where you like it to be.
For example, create a directorey named \texttt{.pipedit} in your 
\texttt{\$HOME} directory and move \texttt{pipedit.cfg} into it.

Edit the config file to your needs, language, PF-keys and such.
\\
\break

\textbf{Three} \\
Edit the profile of your shell, \texttt{.kshrc}, \texttt{.bashrc}, or whatever, 
and define
the environment variable \texttt{\$PIPEDITCFG} and set it to the location of
\texttt{pipedit.cfg} like:

\texttt{PIPEDITCFG=\"/home/myname/.pipedit/pipedit.cfg\" ; export PIPEDITCFG}
\\
\break
And thats it. Log off and log on and try to start pipEdit.

\chapter{Backups}

pipEdit creates backups of the file being edited. Just to be safe…..

The backups are stored in the \texttt{/tmp}
folder and are named, for example, we are editing the file 
\texttt{hello.cob}: \\
\break
\begin{tabular}{l l}
\texttt{hello.cob.grandfather}  & the oldest backup \\
\texttt{hello.cob.father} & the 2nd oldest backup \\
\texttt{hello.cob.son} & the current backup \\
\end{tabular} \\
\break
As you might guess, every editing session, the grandfather is overwritten with the father, the father with the son and the son with the current file before being edited.


\chapter{Keys / Function keys}

\section{Editor}

\begin{tabular}{l l}
F1 & Toggle between short and long (error) message \\
F3 & Quit and save \\
F5 & Repeat find (RFIND) \\
F7 / PgUp & Scroll one page up \\
F8 / PgDown & Scroll one page down \\
F10 & Scroll left \\
F11 & Scroll right \\
F12 & Retrieve \\
\end{tabular}
\\

The function keys are defined in the config file. If you want or must
use other function keys for those actions, edit the config file. \\

For example, the terminal emulation I
use, uses F11 to toggle full screen mode. So scroll right on F11 doesn't work 
very well.  \\

\begin{tabular}{l l}
Arrow up & Cursor up \\
Arrow down & Cursor down \\
Arrow left & Cursor left \\
Arrow right & Cursor right \\
INS & Insert one blank at current position \\
DEL & Deletes one character at current position \\
\end{tabular}

\section{Cancel window}

\begin{tabular}{l l}
F1 & Yes, discard all changes and leave the editor \\
F12 & No, do not discard, stay in editor \\
\end{tabular}

\chapter{Command line commands}

The command line is the line near the top of the screen marked \texttt{Command ===>}, where you can, well..., type commands. \\
Hit \texttt{ENTER} key and the editor will process the typed command. \\

\begin{tabular}{l l}
Save & Save the current file \\
save4macro & Save the current file in a format macros use \\
res / reset & Remove all message lines \\
can / cancel & Cancel editing, must be confirmed again \\
cols & Toggles the display of columns above code \\
l / loc & Locate a line number, move it to top of screen \\
f / find & Find string in text \\
del & DEL ALL NX or DEL ALL X, deletes all eXcluded or non eXcluded lines \\
ver / version & displays the version of pipEdit\\
\end{tabular}



\chapter{Line commands}

Line commands are typed at the line number of a text line. \\ 

\begin{tabular}{l l}
d & Delete line \\
i & Insert blank line \\
r & Repeat line \\
x & eXclude line \\
c & Copy line after/before\\
m & Move line after/before\\
\end{tabular} \\

Every command, except c and m,  takes a number as a parameter.
So \texttt{d3} deletes 3 lines, \texttt{i5} inserts 5 blank lines,
\texttt{r2} repeats the current line two times and \texttt{x7} excludes 7 lines.
\\

Example: \\

\textbf{ \texttt{
\begin{tabular}{l l}
000001  & One \\
d2      & Two \\
000003  & Three \\
000004  & Four \\
000005  & Five \\
\end{tabular} \\
} } \\

The commands c and m need a destination to where copy/move the line, marked with a (after) or b (before) at the line number. \\

Example: \\

\textbf{ \texttt{
\begin{tabular}{l l}
000001  & One \\
c       & Two \\
000003  & Three \\
a       & Four \\
000005  & Five \\
\end{tabular} \\
} }


\chapter{Block commands}

Block commands affect a block of lines (surprise, surprise...), like a block
of lines you want to delete, copy or repeat. \\

A block is marked with the given block commands, so type  \texttt{dd} in the
line number area to mark the beginning of a block you want to delete, and
again \texttt{dd} to mark the end of this block. \\

Example: \\ \\
\texttt{
000005 line5 \\
\textbf{dd}'''' line6 \\
000007 line7 \\
000008 line8 \\
\textbf{dd}'''' line9 \\
000010 line10 \\
} \\
\break
Then hit the \texttt{ENTER} key and the lines will be deleted.

\break
The block commands are: \\

\begin{tabular}{l l}
RR & Repeat a block of lines, duplicate it right after the last line of the block. \\
DD & Delete the block of lines marked with \texttt{dd}  \texttt{dd}. \\
CC & Copy the block after a line marked with \texttt{a} or before a line marked
     with \texttt{b}. \\
XX & eXclude the lines of the block. \\
\end{tabular} \\


\chapter{Macros}

pipEdit supports macros.

A macro is just a program, script, whatever, which is called by pipEdit, 
getting a defined number of parameters and reads and modifies a text file.

\section{Parameters for macros}
\subsection{File name}
This is the name of the temporary file pipEdit writes before calling the macro and reads after the macro is done.

The macro changes this file, like the comp macro, which inserts message lines of the error messages of the compile into the source code.

The format of this file is this (and the result pipEdit reads, must be the same format):

The first 5 lines: \\

\begin{tabular}{l}
\texttt{msg}=Done\\
\texttt{message}=Hex dump done \\
\texttt{cursor\_x}=-1 \\
\texttt{cursor\_y}=-1 \\
\texttt{modified}=0 \\
\end{tabular}
\\
\begin{tabular}{l l}
\textbf{\texttt{msg}} & the short message the macro can set \\
\textbf{\texttt{message}} & the long message (F1) the macro can set \\
\textbf{\texttt{cursor\_x}} & the x position of the cursor the macro can set \\
\textbf{\texttt{cursor\_y}} & the y position of the cursor the macro can set \\
\textbf{\texttt{modified}} & did the macro modify the source code, yes (1) or no (0). This is needed to tell the editor if \\ 
 & the source code must be saved or not. \\
\end{tabular}
\newpage
All lines after the first 5 ones: \\

\begin{tabular}{l l l}
\textbf{Bytes} & \textbf{Type}        & \textbf{Content} \\
0-5            & Line number          & 6 digits line number \\
6              & Line type            & I=inserted line \\
               &                      & M=message\\
               &                      & E=Error \\
               &                      & N=normal text line \\
               &                      & X=eXcluded line \\
7-nnn          & Text                 & The content of the line \\
\end{tabular}


The line number itself is ignored when pipEdit reads the result of the macro again. The line number is used by the comp macro to find the right line in the source code where the error messages are displayed.

\pagebreak
\subsection{Parameter file}
A parameter file, \texttt{filename.par}, looks like this (this is a file for the comp macro):

\#\#\#\#\#\#\#\#\#\#\#\#\#\#\#\#\#\#\#\#\#\#\#\#\#\#\#\#\#\#\#\#\#\#\#\#\#\#\#\#\#\#\#\#\#\#\#\#\#\#

\#\#\# M i c r o f o c u s \#\#\#\#\#\#\#\#\#\#\#\#\#\#\#\#\#\#\#\#\#\#\#\#\#\#\#\#\#\#\#\#\#\#\#\#\#\#\#\#\#\#\#\#\#\#\#\#\#\#

@target=MICROFOCUS

pre=\#!/bin/bash

pre=. /adm/config/basis.prof

pre=export COBCPY=\$HOME/Projects/cpy

\#\#\#----------------------------------

compiler=cob

options=-x -P

binary=hello.bin

listing=hello.lst

movebinto=/home/yourname/yourlocation/Local\_bin/

movelstto=/home/yourname/yourlocation/Listings/

removetmp=.idy .int .cs9Filename

\#\#\#post=rm *.idy *.int *.cs9

\#\#\#\#\#\#\#\#\#\#\#\#\#\#\#\#\#\#\#\#\#\#\#\#\#\#\#\#\#\#\#\#\#\#\#\#\#\#\#\#\#\#\#\#\#\#\#\#\#\#

\#\#\# G n u C o b o l 

\#\#\#\#\#\#\#\#\#\#\#\#\#\#\#\#\#\#\#\#\#\#\#\#\#\#\#\#\#\#\#\#\#\#\#\#\#\#\#\#\#\#\#\#\#\#\#\#\#\#

@target=GNUCOBOL

pre=\#!/bin/bash

compiler=cobc

options=-x -Thello.lst

binary=hello.bin

listing=hello.lst

movebinto=/home/yourname/yourlocation/Binaries/

movelstto=/home/yourname/yourlocation/Listings/

\#\#\#----------------------------------

post=exit

post=\# And thats it


Macros can read this file, in this case the comp macro.

This file specifies how to compile COBOL programs for Microfocus or GnuCobol.

Which compiler is used is specified with the @target= tag.
@target=MICROFOCUS 		or
@target=GNUCOBOL

All lines following the right @target= tag will be used by the comp macro to compile the current source code.

The @target= itself is specified in the pipedit.cfg file.

The pre= lines are written at the beginning of the compile shell script.

The post= lines are written at the end of that script.

Between those lines the comp macro generates code for the compile from the tags

compiler=
name of the COBOL compiler
options=
options for the compiler
binary=
name of the output, the binary
listing=
name of the listing file
movebinto=
Where the result of the compile, the binary, is moved to
movelstto=
where the listing file should be moved to


This is an example of a *.par file for the comp macro.
When you write your own macros, you will create your own parameter files, fitting to the needs of the macros.

\subsection{Filename}
The original name of the file currently being edited.

\subsection{Config file}
This is pipedit.cfg, or however you name it in the environment variable, the macro can read that config file too and use it‘s values.


Those parameters are given to the macro, the macro can use them to read configuration values from the *.par file or from the config file, process and modify the termporary file, and thats it right now.

That is how macros work with pipEdit.

Further plans: Return something like error messages displayed where pipEdit 
displays it‘s own error messages, relocating the cursor and such.

\chapter{comp Macro}
The comp macro is a macro which compiles the (COBOL) source code loaded into pipEdit and displays error messages as message lines right into the source code.



\chapter{hex Macro}

The hex macro inserts message lines of the hex code of every line of the source code into the source code itself.

Type hex to show the hex codes, type res or reset to get rid of them.



\chapter{cs Macro}

The cs macro (Copy to Scratch file) can copy a block of lines into a temporary
file in the \texttt{/tmp} folder and, after leaving the editor and editing a different file,  copy this temporary file back into the editors content. \\


\textbf{Copy a block of lines to a temporary file} \\
Mark a block of lines by typing \texttt{.a} and \texttt{.b} in the line numbers
area of the editor.

Then type \texttt{cs} in the command line to copy the lines into the temporary
file. \\

Example: \\

\texttt{
Command ==> cs \\
000001 line1 \\
000002 line2 \\
000003 line3 \\
000004 line4 \\
000005 line5 \\
\textbf{.a}'''' line6 \\
000007 line7 \\
000008 line8 \\
\textbf{.b}'''' line9 \\
000010 line10 \\
} \\
\break
Then hit the \texttt{ENTER} key and the lines 6, 7, 8 and 9 will be copied
into a temporary file.
\\
\\

\textbf{Copy a block of lines from the temporary file into a different source code} \\

Specify the location where the block of lines should be copied to with
\texttt{a} (after this line) or \texttt{b} (before this line) (and without
a dot ".") in the line numbers area of the editor.

Then type \texttt{cs} in the command line to copy the lines from the temporary
file into the current source code. \\

Example: \\

\texttt{
Command ==> cs \\
000001 line1 \\
000002 line2 \\
000003 line3 \\
000004 line4 \\
000005 line5 \\
\textbf{a}''''' line6 \\
000007 line7 \\
000008 line8 \\
000009 line9 \\
000010 line10 \\
}
\end{document}
